% Programacao Concorrente (ICP-361) (IC/UFRJ)
% Abril de 2024
% Prof.: Silvana Rossetto
% Trabalho: Implementação de uma aplicação concorrente

\documentclass[14]{article}
\usepackage{color,graphicx}
\usepackage{sbc-template}
\usepackage{hyperref}
\hypersetup{
    colorlinks=true,
    linkcolor=blue,
    filecolor=magenta,      
    urlcolor=blue,
    pdfpagemode=FullScreen,
}

\renewcommand{\refname}{Bibliografia}
\begin{document}

\title{{\em Jogo Paralelo da Vida}

{\em \normalsize Gabriel Duarte Soares - 122078892 \and  Henrique Lima Cardoso - 122078397}
\author{ {\Large Relatório Parcial} Programação Concorrente (ICP-361) --- 2024/2}
\address{} 
\date{}}

\maketitle

\section{Descrição do problema geral}{
Descrever o problema escolhido:

O projeto consiste na simulação de autômatos celulares, um 
\href{https://en.wikipedia.org/wiki/Zero-player_game}{Jogo de Zero Jogadores} onde 
o espaço é dividido em células (ou "automatos") que mudam de estado ao longo do tempo, 
seguindo um conjunto de regras específicas. Nós simularemos o 
\href{https://en.wikipedia.org/wiki/Conway's_Game_of_Life}{Game of life}
elaborado pelo  \href{https://en.wikipedia.org/wiki/John_Horton_Conway}{John Conway}, 
onde cada célula representa uma célula "viva" ou "morta", e seu estado em cada passo do tempo depende do 
estado das células vizinhas.

Na versão sequencial, a simulação de autômatos celulares funciona da seguinte forma:
Um grid bidimensional (ou tridimensional, para uma versão mais avançada) é inicializado, onde cada célula possui um estado.
A cada frame, o programa avalia o estado de cada célula e de suas vizinhas, com base em regras predefinidas.
Isso é repetido por um número de iterações definido pelo usuário - podendo ser infinito - gerando uma sequência de grids que 
representam a evolução do sistema ao longo do tempo.

Dados entrada:
\begin{itemize}
    \item Configuração Inicial das células, representado por um array que guarda o estado de cada uma delas.
    \item Dimensão do conjunto de células - se é bi, tri ou n-dimensional, e o tamanho do universo.
    \item Números de iterações: por quanto tempo o autômato deve rodar, ou se deve rodar infinitamente.
    \item Regras de Atualização: Conjunto de regras que determinam como uma célula muda de estado com base 
nos estados das células vizinhas. (No jogo do Conway, já é determinado)
    \item (Opcionais) Velocidade e Bufferização
\end{itemize}

O programa deve mostrar a evolução do jogo ao longo do tempo. Um registro das configurações do grid em cada passo,
que pode ser exibido como uma sequência de imagens ou como uma animação. (Teremos que decidir a melhor forma de fazer isso)

}

%-----------------------------------------------------------------------
\section{Projeto da solução concorrente}
{Descrever o projeto da solução concorrente para o problema: 

Ao implementar uma solução concorrente para a simulação de autômatos celulares, a tarefa principal – 
atualização do grid ao longo do tempo – pode ser dividida entre diferentes threads ou rotinas paralelas.
Acelerando significativamente o programa. Seria muito interessante, também, usar GPU's para um paralelismo
ainda maior - isso ficará a mercê da nossa capacidade :).

Inicialmente, dividiremos os espaço da maneira mais simples que conheçemos - assim como nas primeiras aulas - designaremos
blocos para cada thread, de forma a isolar o espaço de atuação de cada uma delas. Nesse quesito, esse programa é 
\href{https://en.wikipedia.org/wiki/Embarrassingly_parallel}{embaraçosamente paralelo}.
Posteriormente, podemos estudar formas mais avançadas de otimização do Jogo da Vida e ver as possibilidades de paralelismo nelas.

%-----------------------------------------------------------------------
\section{Casos de teste de corretude e desempenho}
{Descrever como o programa será testado:

Para avaliar o desempenho, os testes devem medir o tempo de execução da solução concorrente para diferentes tamanhos de grids 
e números de threads. Espera-se que o tempo de execução diminua  significativamente com o aumento do número de threads.

Como o output do programa é uma sequência de tabelas - ou estado do universo. Quando testarmos a corretude do programa,
teremos que comparar a solução paralela com a solução sequencial para cada frame. Isso será fácil de fazer, pois não envolve
a parte gráfica - da visualização do programa.

Para testar o desempenho, olharemos para grids com dimensões distintas (pequeno até enorme).
Podemos querer saber também a interferência na densidade de células vivas inicialmente.

Espera-se que o programa tenha quase 100\% de eficiência! Isto é: o speedup é linear em relação ao número de threads.
Mas em grids pequenos, a sobrecarga de comunicação pode limitar os ganhos, diminuindo a eficiência.

Se tivermos a oportunidade de implementar técnicas mais avançadas para a atualização do autômato,
será divertido também observar qual é o impacto da paralelização.

}


%-----------------------------------------------------------------------
\section{Referências bibliográficas}
{   
    \begin{itemize}
        \item \url{https://en.wikipedia.org/wiki/Conway's_Game_of_Life}
        \item \url{https://en.wikipedia.org/wiki/Embarrassingly_parallel}
        \item \url{https://rbeaulieu.github.io/3DGameOfLife/3DGameOfLife.html?}
        \item \url{https://kodub.itch.io/game-of-life-3d}
        \item \url{https://en.wikipedia.org/wiki/John_Horton_Conway}
        \item \url{https://en.wikipedia.org/wiki/Zero-player_game}
        \item Slides da professora Silvana Rossetto
    \end{itemize}
}

%-------------------------------------
%\bibliographystyle{unsrt}
%\bibliography{./bibfile}

%-----------------------------------------------------------
\end{document}
